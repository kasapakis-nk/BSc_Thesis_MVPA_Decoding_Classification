\thispagestyle{plain}
\vspace*{\fill}
\begin{center}
    \LARGE
    \textit{\textbf{Abstract}}
        
    \vspace{0.4cm}
    \large
    \textbf{Decoding and Classification of Category-Specific Visual Stimuli in the Fusiform Face Area Using fMRI Data and Machine Learning}
        
    \vspace{0.4cm}
    Kasapakis Nikolaos
\end{center}
\normalsize

\vspace{0.9cm}

This thesis examines the use of \gls{fMRI} data from the \gls{HCP} to analyze neural systems involved in the processing of category-specific visual stimuli. It focuses on the \gls{WM} task within a broad \gls{fMRI} paradigm designed to explore various neural domains including visual, language, and decision-making systems. Specifically, the study examines brain activation patterns in the \gls{FFA} related to different stimulus categories such as faces, places, tools, and body parts. Employing a pipeline of preprocessing steps followed by \gls{MVPA}, the research assesses distributed patterns of voxel activation allowing for complex and detailed analyses of cognitive states. Data is processed through a \	gls{SVM} classification script developed for this study, aiming to differentiate between faces and other stimulus categories based solely on \gls{fMRI} data. The thesis also addresses classification methodologies, with a particular focus on optimizing classifier accuracy by adjusting parameters like the number of data chunks, fold counts for cross-validation, and subject counts. The findings offer insights into the distinct brain activation patterns associated with different stimuli in the \gls{FFA}, contributing to our understanding of neural mechanisms in cognitive processes and practical applications for these insights.

\vspace*{\fill}

%\footnote{Temporal refers to time-related when used in the context of electromagnetics.}
%Creates a footnote in the same page with that message.