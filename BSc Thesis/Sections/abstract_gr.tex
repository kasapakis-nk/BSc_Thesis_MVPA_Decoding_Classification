\thispagestyle{plain}
\vspace*{\fill}
\begin{center}
    \LARGE
    \textit{\textbf{Περίληψη}}
        
    \vspace{0.4cm}
    \large
    \textbf{Αποκωδικοποίηση και Κατηγοριοποίηση Κατηγορικών Οπτικών Ερεθισμάτων στην Περιοχή του Fusiform Face Area με Χρήση Δεδομένων fMRI και Μηχανικής Μάθησης}
        
    \vspace{0.4cm}
    Κασαπάκης Νικόλαος
\end{center}
\normalsize

\vspace{0.9cm}

Αυτή η εργασία εξετάζει τη χρήση δεδομένων λειτουργικής μαγνητικής τομογραφίας (\acrshort{fMRI}) από το \acrshort{HCP} για την ανάλυση των νευρωνικών συστημάτων που εμπλέκονται στην επεξεργασία κατηγορικών οπτικών ερεθισμάτων. Επικεντρώνεται στη δοκιμασία Λειτουργικής Μνήμης (\acrshort{WM}) μέσα σε ένα ευρύτερο πείραμα fMRI που έχει σχεδιαστεί για να εξερευνήσει διάφορους νευρωνικούς τομείς, συμπεριλαμβανομένων των συστημάτων όρασης, γλώσσας και λήψης αποφάσεων. Συγκεκριμένα, η μελέτη εξετάζει τα μοτίβα ενεργοποίησης του εγκεφάλου στην περιοχή \gls{FFA} που σχετίζονται με διάφορες κατηγορίες οπτικών ερεθισμάτων, όπως πρόσωπα, τοποθεσίες, εργαλεία και μέρη του σώματος. Χρησιμοποιώντας μια σειρά προεπεξεργαστικών βημάτων ακολουθούμενη από Ανάλυση Πολυδιάστατων Μοτίβων (\acrshort{MVPA}), η έρευνα αξιολογεί κατανεμημένα μοτίβα ενεργοποίησης, επιτρέποντας σύνθετες και λεπτομερείς αναλύσεις των γνωστικών καταστάσεων. Τα δεδομένα επεξεργάζονται μέσω ενός υπολογιστικού προγράμμα\-τος κατηγοριοποίησης \acrshort{SVM} που αναπτύχθηκε για αυτή τη μελέτη, με στόχο τη διάκριση μεταξύ προσώπων και άλλων κατηγοριών ερεθισμάτων βασισμένων αποκλειστικά σε δεδομένα \acrshort{fMRI}. Η εργασία εξετάζει επίσης μεθοδολογίες ταξινόμησης, δίνοντας ιδιαίτερη έμφαση στη                                βελτιστοποίηση της ακρίβειας του κατηγοριοποιητή με την προσαρμογή παραμέτρων όπως ο αριθμός των ανεξάρτητων τμημάτων δεδομένων, ο αριθμός επαναληπτικών διασταυρώσεων για επικύρωση και ο αριθμός των υποκειμένων των οποίων τα δεδομένα εμπλέκονται. Τα ευρήματα προσφέρουν πληροφορίες για τα διακριτά μοτίβα ενεργοποίησης του εγκεφάλου που σχετίζονται με διάφορα ερεθίσματα στην \acrshort{FFA}, συμβάλλοντας στην κατανόησή μας για τους νευρωνικούς μηχανισμούς στις γνωστικές διαδικασίες και στην ανάπτυξη πιθανών πρακτικών εφαρμογών μέσω αυτών των ευρημάτων.

\vspace*{\fill}

