\pagebreak
\chapter{Conclusions}
\label{sec:conclusions}

Significant progress has been made in brain functional mapping over the past few decades. This thesis has shown the feasibility and efficiency of using \gls{MVPA} in both the \gls{FFA} and \gls{PPA} to decode category-specific neural patterns. As more discoveries link specific brain functions to distinct regions, the methodology presented here offers a framework that can be adapted to a wide range of research questions, both within and beyond the realm of visual processing. The thorough analysis of baseline signals, which integrates traditional univariate methods with advanced multivariate techniques, sets a standard for future research. It also lays the groundwork for the development of more sophisticated tools, enhancing both the quality of results and the efficiency of resource use in the development process.

This research also offers practical implications for the development of more efficient and reliable brain-computer interfaces \gls{BCI}s. The findings demonstrate that accurate neural decoding can be achieved with a reduced number of folds and subjects, significantly lowering the computational cost and time required. This is particularly relevant for clinical applications, where quick and accurate neural decoding could enhance the effectiveness of neurofeedback and other therapeutic interventions.

Looking ahead, future research should explore several avenues to build upon and refine the findings of this study. First, utilizing data from large-scale functional mapping projects, similar in scope to the \gls{HCP}, could significantly enhance the generalizability, reliability, and statistical power of the results. Additionally, longitudinal studies would be valuable in examining how neural representations evolve over time or with training, offering insights into the plasticity of neural coding. 

Another promising direction is to apply \gls{MVPA} to other brain regions involved in visual processing, such as the \gls{OFA} and the \gls{LOC}. Expanding the analysis to these areas could provide a more comprehensive understanding of the neural mechanisms underlying face and place recognition, potentially revealing the hierarchical processing stages across different cortical regions.

Lastly, integrating \gls{MVPA} with other neuroimaging techniques like \gls{EEG} or \gls{MEG} could shed light on the temporal dynamics of category-specific processing. This multimodal approach would bridge the gap between the high spatial resolution of \gls{fMRI} and the superior temporal resolution of \gls{EEG}/\gls{MEG}, offering a more complete and nuanced picture of neural processing.

Despite the promising results, this study is not without its limitations. A notable constraint is the relatively small sample size, which may impact the statistical power and generalizability of the findings. Additionally, the study primarily focused on two specific brain regions, the \gls{FFA} and \gls{PPA}. While these areas are key to face and place recognition, and were chosen due to the strong established connections between them and their associated functions, other relevant brain areas were not explored. This narrow focus could limit the broader applicability of the \gls{MVPA} techniques and the generalization of these classification methods to more complex cognitive processes.

Finally, the study's reliance on data from the \gls{HCP} may introduce some bias, as this dataset is preprocessed and may not fully represent the variability and complexity of raw \gls{fMRI} data. Future studies should aim to validate these findings using raw, unprocessed data to ensure the robustness of the conclusions.

In conclusion, this thesis not only advances our understanding of the neural mechanisms underlying category-specific visual processing but also lays the groundwork for future research that could lead to significant developments in both theoretical and applied neuroscience.