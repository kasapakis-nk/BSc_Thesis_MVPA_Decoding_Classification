\newacronym{MVPA}{MVPA}{Multi-Variate Pattern Analysis}
\newacronym{UPA}{UPA}{Univariate Pattern Analysis}
\newacronym{SVM}{SVM}{Support Vector Machine}
\newacronym{ROI}{ROI}{Region of Interest}
\newacronym{ROIs}{ROIs}{Regions of Interest}
\newacronym{BOLD}{BOLD}{Blood-Oxygen-Level-Dependent}
\newacronym{HbR}{HbR}{deoxyhemoglobin}
\newacronym{HbT}{HbT}{total hemoglobin}
\newacronym{HbO}{HbO}{Oxyhemoglobin}
\newacronym{Hb}{Hb}{hemoglobin}
\newacronym{OEF}{OEF}{Oxygen Extraction Fraction}
\newacronym{HCP}{HCP}{Human Connectome Project}
\newacronym{WM}{WM}{Working Memory}
\newacronym{ITI}{ITI}{inter-task interval}
\newacronym{CSF}{CSF}{Cerebrospinal Fluid}
\newacronym{RF}{RF}{radio frequency}
\newacronym{CBF}{CBF}{Cerebral Blood Flow}
\newacronym{ASL}{ASL}{Arterial Spin Labeling}
\newacronym{USA}{USA}{United States of America}
\newacronym{T1w}{T1w}{T1-weighted}
\newacronym{T2w}{T2w}{T2-weighted}
\newacronym{MZ}{MZ}{monozygotic}
\newacronym{DZ}{DZ}{dizygotic}
\newacronym{IQ}{IQ}{intelligence quotient}
\newacronym{T-MEG}{T-MEG}{task-evoked MEG}
\newacronym{EV}{EV}{explanatory variable}
\newacronym{COPE}{COPE}{contrast of explanatory variables}
\newacronym{SVMs}{SVMs}{Support Vector Machines}
\newacronym{PCA}{PCA}{Principal Component Analysis}
\newacronym{ICA}{ICA}{Independent Component Analysis}
\newacronym{PDF}{PDF}{probability density function}
\newacronym{HRF}{HRF}{Hemodynamic Response Function}
\newacronym{GLM}{GLM}{General Linear Model}
\newacronym{NIfTI}{NIfTI}{Neuroimaging Informatics Technology Initiative}
\newacronym{LR}{LR}{Left to Right}
\newacronym{RL}{RL}{Right to Left}
\newacronym{FEAT}{FEAT}{FMRIB's Expert Analysis Tool}
\newacronym{3D MPRAGE}{3D MPRAGE}{three-dimensional magnetization-prepared rapid gradient-echo imaging}
\newacronym{FOV}{FOV}{Field of View}
\newacronym{BW}{BW}{Bandwidth}
\newacronym{ES}{ES}{Echo Spacing}
\newacronym{BCI}{BCI}{Brain-Computer Interface}

% fMRI Acronyms
\newacronym{r-fMRI}{r-fMRI}{resting-state fMRI}
\newacronym{t-fMRI}{t-fMRI}{task-evoked fMRI}
\newacronym{dMRI}{dMRI}{diffusion imaging}
\newacronym{fMRI}{fMRI}{Functional Magnetic Resonance Imaging}
\newacronym{MRI}{MRI}{Magnetic Resonance Imaging}
\newacronym{MR}{MR}{Magnetic Resonance}
\newacronym{NMR}{NMR}{Nuclear Magnetic Resonance}

% Imaging Techniques Acronyms
\newacronym{PET}{PET}{Positron Emission Tomography}
\newacronym{NIRS}{NIRS}{Near Infrared Spectroscopy}
\newacronym{MEG}{MEG}{Magnetoencephalogram}
\newacronym{EEG}{EEG}{Electroencephalography}

% Brain Anatomy Acronyms
\newacronym{LGN}{LGN}{Lateral Geniculate Nucleus}
\newacronym{MT}{MT}{Middle Temporal visual area}
\newacronym{IT}{IT}{Inferior Temporal cortex}
\newacronym{FFA}{FFA}{Fusiform Face Area}
\newacronym{PPA}{PPA}{Parahippocampal Place Area}
\newacronym{LOC}{LOC}{Lateral Occipital Cortex}
\newacronym{EBA}{EBA}{Extrastriate Body Area}
\newacronym{FBA}{FBA}{Fusiform Body Area}
\newacronym{FG}{FG}{Fusiform Gyrus}
\newacronym{fSTS}{fSTS}{Superior Temporal Sulcus}
\newacronym{OFA}{OFA}{Occipital Face Area}
\newacronym{MOG}{MOG}{Middle Occipital Gyrus}
\newacronym{IOG}{IOG}{Inferior Occipital Gyrus}
\newacronym{LOS}{LOS}{Lateral Occipital Sulcus}

% Establishments Acronyms
\newacronym{WashU}{WashU}{Washington University}
\newacronym{UMinn}{UMinn}{University of Minnesota}
\newacronym{SLU}{SLU}{Saint Louis University}
\newacronym{NIH}{NIH}{National Institutes of Health}

% Glossary Entries
\newglossaryentry{V1}
{
	name = {V1},
	description = {Visual area V1, the striate cortex or primary visual cortex.\newline}
}

\newglossaryentry{V2}
{
	name = {V2},
	description = {Visual area V2, or secondary visual cortex, also called prestriate cortex.\newline}
}

\newglossaryentry{V3}
{
	name = {V3},
	description = {Visual area V3, which communicates directly with the respective dorsal and ventral subsystems of V2. It is less well-defined compared to other areas of the visual cortex.\newline}
}

\newglossaryentry{V4}
{
	name = {V4},
	description = {Visual area V4, a mid-tier cortical area in the ventral visual pathway.\newline}
}

\newglossaryentry{blob cells}
{
	name = {blob cells},
	description = {V1 cells that resemble kLGN neurons. They are monocular, color sensitive, characterized by small, concentric receptive fields and are found in clusters, hence the name.\newline}
}

\newglossaryentry{interblob cells}
{
	name = {interblob cells},
	description = {V1 cells, the majority of which are binocular, not color sensitive, characterized by elongated receptive fields, exhibit ocular dominance and orientation specificity, while they are found around the clusters of V1 blob cells.\newline}
}

\newglossaryentry{gyromagnetic ratio}
{
	name = {gyromagnetic ratio},
	description = {The gyromagnetic ratio, a constant specific to each different nucleus.\newline}
}

\newglossaryentry{TE}
{
	name = {Time of Echo},
	description = {The time between the delivery of the RF pulse and the receipt of the echo signal.\newline}
}

\newglossaryentry{TR}
{
	name = {TR},
	description = {The amount of time that passes between consecutive acquired brain volumes.\newline}
}

\newglossaryentry{flip angle}
{
	name = {flip angle},
	description = {The amount or rotation that net magnetization experiences during application of a RF pulse.\newline}
}

\newglossaryentry{connectomics}
{
	name = {connectomics},
	description = {The production and study of connectomes: comprehensive maps of connections within an organism's nervous system.\newline}
}

\newglossaryentry{2C}
{
	name = {2C},
	description = {The FEAT analysis that produced 2 chunks per subject, with a total of 40 chunks.\newline}
}

\newglossaryentry{4C}
{
	name = {4C},
	description = {The FEAT analysis that produced 4 chunks per subject, with a total of 80 chunks.\newline}
}